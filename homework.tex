% Options for packages loaded elsewhere
\PassOptionsToPackage{unicode}{hyperref}
\PassOptionsToPackage{hyphens}{url}
%
\documentclass[
]{article}
\usepackage{lmodern}
\usepackage{amssymb,amsmath}
\usepackage{ifxetex,ifluatex}
\ifnum 0\ifxetex 1\fi\ifluatex 1\fi=0 % if pdftex
  \usepackage[T1]{fontenc}
  \usepackage[utf8]{inputenc}
  \usepackage{textcomp} % provide euro and other symbols
\else % if luatex or xetex
  \usepackage{unicode-math}
  \defaultfontfeatures{Scale=MatchLowercase}
  \defaultfontfeatures[\rmfamily]{Ligatures=TeX,Scale=1}
\fi
% Use upquote if available, for straight quotes in verbatim environments
\IfFileExists{upquote.sty}{\usepackage{upquote}}{}
\IfFileExists{microtype.sty}{% use microtype if available
  \usepackage[]{microtype}
  \UseMicrotypeSet[protrusion]{basicmath} % disable protrusion for tt fonts
}{}
\makeatletter
\@ifundefined{KOMAClassName}{% if non-KOMA class
  \IfFileExists{parskip.sty}{%
    \usepackage{parskip}
  }{% else
    \setlength{\parindent}{0pt}
    \setlength{\parskip}{6pt plus 2pt minus 1pt}}
}{% if KOMA class
  \KOMAoptions{parskip=half}}
\makeatother
\usepackage{xcolor}
\IfFileExists{xurl.sty}{\usepackage{xurl}}{} % add URL line breaks if available
\IfFileExists{bookmark.sty}{\usepackage{bookmark}}{\usepackage{hyperref}}
\hypersetup{
  pdftitle={Homework},
  hidelinks,
  pdfcreator={LaTeX via pandoc}}
\urlstyle{same} % disable monospaced font for URLs
\usepackage[margin=1in]{geometry}
\usepackage{color}
\usepackage{fancyvrb}
\newcommand{\VerbBar}{|}
\newcommand{\VERB}{\Verb[commandchars=\\\{\}]}
\DefineVerbatimEnvironment{Highlighting}{Verbatim}{commandchars=\\\{\}}
% Add ',fontsize=\small' for more characters per line
\usepackage{framed}
\definecolor{shadecolor}{RGB}{248,248,248}
\newenvironment{Shaded}{\begin{snugshade}}{\end{snugshade}}
\newcommand{\AlertTok}[1]{\textcolor[rgb]{0.94,0.16,0.16}{#1}}
\newcommand{\AnnotationTok}[1]{\textcolor[rgb]{0.56,0.35,0.01}{\textbf{\textit{#1}}}}
\newcommand{\AttributeTok}[1]{\textcolor[rgb]{0.77,0.63,0.00}{#1}}
\newcommand{\BaseNTok}[1]{\textcolor[rgb]{0.00,0.00,0.81}{#1}}
\newcommand{\BuiltInTok}[1]{#1}
\newcommand{\CharTok}[1]{\textcolor[rgb]{0.31,0.60,0.02}{#1}}
\newcommand{\CommentTok}[1]{\textcolor[rgb]{0.56,0.35,0.01}{\textit{#1}}}
\newcommand{\CommentVarTok}[1]{\textcolor[rgb]{0.56,0.35,0.01}{\textbf{\textit{#1}}}}
\newcommand{\ConstantTok}[1]{\textcolor[rgb]{0.00,0.00,0.00}{#1}}
\newcommand{\ControlFlowTok}[1]{\textcolor[rgb]{0.13,0.29,0.53}{\textbf{#1}}}
\newcommand{\DataTypeTok}[1]{\textcolor[rgb]{0.13,0.29,0.53}{#1}}
\newcommand{\DecValTok}[1]{\textcolor[rgb]{0.00,0.00,0.81}{#1}}
\newcommand{\DocumentationTok}[1]{\textcolor[rgb]{0.56,0.35,0.01}{\textbf{\textit{#1}}}}
\newcommand{\ErrorTok}[1]{\textcolor[rgb]{0.64,0.00,0.00}{\textbf{#1}}}
\newcommand{\ExtensionTok}[1]{#1}
\newcommand{\FloatTok}[1]{\textcolor[rgb]{0.00,0.00,0.81}{#1}}
\newcommand{\FunctionTok}[1]{\textcolor[rgb]{0.00,0.00,0.00}{#1}}
\newcommand{\ImportTok}[1]{#1}
\newcommand{\InformationTok}[1]{\textcolor[rgb]{0.56,0.35,0.01}{\textbf{\textit{#1}}}}
\newcommand{\KeywordTok}[1]{\textcolor[rgb]{0.13,0.29,0.53}{\textbf{#1}}}
\newcommand{\NormalTok}[1]{#1}
\newcommand{\OperatorTok}[1]{\textcolor[rgb]{0.81,0.36,0.00}{\textbf{#1}}}
\newcommand{\OtherTok}[1]{\textcolor[rgb]{0.56,0.35,0.01}{#1}}
\newcommand{\PreprocessorTok}[1]{\textcolor[rgb]{0.56,0.35,0.01}{\textit{#1}}}
\newcommand{\RegionMarkerTok}[1]{#1}
\newcommand{\SpecialCharTok}[1]{\textcolor[rgb]{0.00,0.00,0.00}{#1}}
\newcommand{\SpecialStringTok}[1]{\textcolor[rgb]{0.31,0.60,0.02}{#1}}
\newcommand{\StringTok}[1]{\textcolor[rgb]{0.31,0.60,0.02}{#1}}
\newcommand{\VariableTok}[1]{\textcolor[rgb]{0.00,0.00,0.00}{#1}}
\newcommand{\VerbatimStringTok}[1]{\textcolor[rgb]{0.31,0.60,0.02}{#1}}
\newcommand{\WarningTok}[1]{\textcolor[rgb]{0.56,0.35,0.01}{\textbf{\textit{#1}}}}
\usepackage{graphicx,grffile}
\makeatletter
\def\maxwidth{\ifdim\Gin@nat@width>\linewidth\linewidth\else\Gin@nat@width\fi}
\def\maxheight{\ifdim\Gin@nat@height>\textheight\textheight\else\Gin@nat@height\fi}
\makeatother
% Scale images if necessary, so that they will not overflow the page
% margins by default, and it is still possible to overwrite the defaults
% using explicit options in \includegraphics[width, height, ...]{}
\setkeys{Gin}{width=\maxwidth,height=\maxheight,keepaspectratio}
% Set default figure placement to htbp
\makeatletter
\def\fps@figure{htbp}
\makeatother
\setlength{\emergencystretch}{3em} % prevent overfull lines
\providecommand{\tightlist}{%
  \setlength{\itemsep}{0pt}\setlength{\parskip}{0pt}}
\setcounter{secnumdepth}{-\maxdimen} % remove section numbering

\title{Homework}
\author{}
\date{\vspace{-2.5em}}

\begin{document}
\maketitle

\hypertarget{exercise-2}{%
\section{Exercise 2}\label{exercise-2}}

\hypertarget{section}{%
\subsection{1.}\label{section}}

\hypertarget{air-quality-data-set}{%
\subsubsection{Air Quality Data Set}\label{air-quality-data-set}}

Source: \url{https://archive.ics.uci.edu/ml/datasets/Air+quality}

\begin{verbatim}
## Warning: package 'MTS' was built under R version 3.6.3
\end{verbatim}

First we read the data and do some preprocessing.

\begin{Shaded}
\begin{Highlighting}[]
\NormalTok{air <-}\StringTok{ }\KeywordTok{read.csv}\NormalTok{(}\StringTok{"Data/AirQualityUCI.csv"}\NormalTok{, }\DataTypeTok{header =} \OtherTok{TRUE}\NormalTok{, }\DataTypeTok{sep =} \StringTok{";"}\NormalTok{)}
\NormalTok{air <-}\StringTok{ }\NormalTok{air[,}\KeywordTok{c}\NormalTok{(}\DecValTok{1}\NormalTok{,}\DecValTok{2}\NormalTok{,}\DecValTok{3}\NormalTok{,}\DecValTok{6}\NormalTok{,}\DecValTok{13}\NormalTok{)]}
\NormalTok{air}\OperatorTok{$}\NormalTok{Date <-}\StringTok{ }\KeywordTok{as.Date}\NormalTok{(air}\OperatorTok{$}\NormalTok{Date, }\DataTypeTok{format=}\StringTok{"%d/%m/%Y"}\NormalTok{)}
\NormalTok{air}\OperatorTok{$}\NormalTok{Time <-}\StringTok{ }\KeywordTok{as.numeric}\NormalTok{(air}\OperatorTok{$}\NormalTok{Time)}\OperatorTok{-}\DecValTok{2}
\NormalTok{air}\OperatorTok{$}\NormalTok{CO.GT. <-}\StringTok{ }\KeywordTok{as.numeric}\NormalTok{(air}\OperatorTok{$}\NormalTok{CO.GT.)}
\NormalTok{air}\OperatorTok{$}\NormalTok{T <-}\StringTok{ }\KeywordTok{as.numeric}\NormalTok{(air}\OperatorTok{$}\NormalTok{T)}
\NormalTok{air}\OperatorTok{$}\NormalTok{C6H6.GT. <-}\StringTok{ }\KeywordTok{as.numeric}\NormalTok{(air}\OperatorTok{$}\NormalTok{C6H6.GT.)}
\NormalTok{n =}\StringTok{ }\KeywordTok{dim}\NormalTok{(air)[}\DecValTok{1}\NormalTok{]}
\end{Highlighting}
\end{Shaded}

We will look at 3 different versions of the data: Original,
log-transformation, series of differences of the log-transformation.

\begin{Shaded}
\begin{Highlighting}[]
\NormalTok{Y_orig <-}\StringTok{ }\NormalTok{air[,}\DecValTok{3}\OperatorTok{:}\DecValTok{5}\NormalTok{]}
\NormalTok{Y_log=}\KeywordTok{log}\NormalTok{(air[,}\DecValTok{3}\OperatorTok{:}\DecValTok{5}\NormalTok{])}
\NormalTok{Y_rate <-}\StringTok{ }\NormalTok{Y_log[}\DecValTok{2}\OperatorTok{:}\NormalTok{n,] }\OperatorTok{-}\StringTok{ }\NormalTok{Y_log[}\DecValTok{1}\OperatorTok{:}\NormalTok{(n}\DecValTok{-1}\NormalTok{),]}
\NormalTok{Y_rate <-}\StringTok{ }\DecValTok{100}\OperatorTok{*}\NormalTok{Y_rate}
\end{Highlighting}
\end{Shaded}

Original:

\begin{Shaded}
\begin{Highlighting}[]
\KeywordTok{par}\NormalTok{(}\DataTypeTok{mfrow=}\KeywordTok{c}\NormalTok{(}\DecValTok{2}\NormalTok{,}\DecValTok{3}\NormalTok{))}
\KeywordTok{plot}\NormalTok{(air}\OperatorTok{$}\NormalTok{Date,Y_orig[,}\DecValTok{1}\NormalTok{],}\DataTypeTok{type=}\StringTok{"l"}\NormalTok{,}\DataTypeTok{xlab=}\StringTok{""}\NormalTok{,}\DataTypeTok{ylab=}\StringTok{"Log"}\NormalTok{,}\DataTypeTok{main=}\StringTok{"CO.GT"}\NormalTok{)}
\KeywordTok{plot}\NormalTok{(air}\OperatorTok{$}\NormalTok{Date,Y_orig[,}\DecValTok{2}\NormalTok{],}\DataTypeTok{type=}\StringTok{"l"}\NormalTok{,}\DataTypeTok{xlab=}\StringTok{""}\NormalTok{,}\DataTypeTok{ylab=}\StringTok{"Log"}\NormalTok{,}\DataTypeTok{main=}\StringTok{"C6H6.GT"}\NormalTok{)}
\KeywordTok{plot}\NormalTok{(air}\OperatorTok{$}\NormalTok{Date,Y_orig[,}\DecValTok{3}\NormalTok{],}\DataTypeTok{type=}\StringTok{"l"}\NormalTok{,}\DataTypeTok{xlab=}\StringTok{""}\NormalTok{,}\DataTypeTok{ylab=}\StringTok{"Log"}\NormalTok{,}\DataTypeTok{main=}\StringTok{"T"}\NormalTok{)}
\KeywordTok{acf}\NormalTok{(Y_orig[,}\DecValTok{1}\NormalTok{],}\DataTypeTok{main=}\StringTok{""}\NormalTok{)}
\KeywordTok{acf}\NormalTok{(Y_orig[,}\DecValTok{2}\NormalTok{],}\DataTypeTok{main=}\StringTok{""}\NormalTok{)}
\KeywordTok{acf}\NormalTok{(Y_orig[,}\DecValTok{3}\NormalTok{],}\DataTypeTok{main=}\StringTok{""}\NormalTok{)}
\end{Highlighting}
\end{Shaded}

\includegraphics{homework_files/figure-latex/unnamed-chunk-4-1.pdf} We
see that the original data is not stationary.

Log-Transformation and Rates:

\begin{Shaded}
\begin{Highlighting}[]
\KeywordTok{par}\NormalTok{(}\DataTypeTok{mfrow=}\KeywordTok{c}\NormalTok{(}\DecValTok{3}\NormalTok{,}\DecValTok{3}\NormalTok{))}
\KeywordTok{plot}\NormalTok{(air}\OperatorTok{$}\NormalTok{Date,Y_log[,}\DecValTok{1}\NormalTok{],}\DataTypeTok{type=}\StringTok{"l"}\NormalTok{,}\DataTypeTok{xlab=}\StringTok{""}\NormalTok{,}\DataTypeTok{ylab=}\StringTok{"Log"}\NormalTok{,}\DataTypeTok{main=}\StringTok{"CO.GT"}\NormalTok{)}
\KeywordTok{plot}\NormalTok{(air}\OperatorTok{$}\NormalTok{Date,Y_log[,}\DecValTok{2}\NormalTok{],}\DataTypeTok{type=}\StringTok{"l"}\NormalTok{,}\DataTypeTok{xlab=}\StringTok{""}\NormalTok{,}\DataTypeTok{ylab=}\StringTok{"Log"}\NormalTok{,}\DataTypeTok{main=}\StringTok{"C6H6.GT"}\NormalTok{)}
\KeywordTok{plot}\NormalTok{(air}\OperatorTok{$}\NormalTok{Date,Y_log[,}\DecValTok{3}\NormalTok{],}\DataTypeTok{type=}\StringTok{"l"}\NormalTok{,}\DataTypeTok{xlab=}\StringTok{""}\NormalTok{,}\DataTypeTok{ylab=}\StringTok{"Log"}\NormalTok{,}\DataTypeTok{main=}\StringTok{"T"}\NormalTok{)}
\KeywordTok{plot}\NormalTok{(air}\OperatorTok{$}\NormalTok{Date[}\DecValTok{2}\OperatorTok{:}\NormalTok{n],Y_rate[,}\DecValTok{1}\NormalTok{],}\DataTypeTok{type=}\StringTok{"l"}\NormalTok{,}\DataTypeTok{xlab=}\StringTok{""}\NormalTok{,}\DataTypeTok{ylab=}\StringTok{"Rate"}\NormalTok{)}
\KeywordTok{plot}\NormalTok{(air}\OperatorTok{$}\NormalTok{Date[}\DecValTok{2}\OperatorTok{:}\NormalTok{n],Y_rate[,}\DecValTok{2}\NormalTok{],}\DataTypeTok{type=}\StringTok{"l"}\NormalTok{,}\DataTypeTok{xlab=}\StringTok{""}\NormalTok{,}\DataTypeTok{ylab=}\StringTok{"Rate"}\NormalTok{)}
\KeywordTok{plot}\NormalTok{(air}\OperatorTok{$}\NormalTok{Date[}\DecValTok{2}\OperatorTok{:}\NormalTok{n],Y_rate[,}\DecValTok{3}\NormalTok{],}\DataTypeTok{type=}\StringTok{"l"}\NormalTok{,}\DataTypeTok{xlab=}\StringTok{""}\NormalTok{,}\DataTypeTok{ylab=}\StringTok{"Rate"}\NormalTok{)}
\KeywordTok{acf}\NormalTok{(Y_rate[,}\DecValTok{1}\NormalTok{],}\DataTypeTok{main=}\StringTok{""}\NormalTok{)}
\KeywordTok{acf}\NormalTok{(Y_rate[,}\DecValTok{2}\NormalTok{],}\DataTypeTok{main=}\StringTok{""}\NormalTok{)}
\KeywordTok{acf}\NormalTok{(Y_rate[,}\DecValTok{3}\NormalTok{],}\DataTypeTok{main=}\StringTok{""}\NormalTok{)}
\end{Highlighting}
\end{Shaded}

\includegraphics{homework_files/figure-latex/unnamed-chunk-5-1.pdf}

\end{document}
